% ===============================================================
%            LaTeX Setting for IEEE Newsletter
%
%    Author  :  Bei Yu@CUHK
%    Email   :  byu@cse.cuhk.edu.hk
%    Updated :  Oct. 2015
%
% ===============================================================

\setlength\topmargin{-48pt}      % Top margin
\setlength\headheight{0pt}       % Header height
\setlength\textwidth{7.0in}      % Text width
\setlength\textheight{9.5in}     % Text height
\setlength\oddsidemargin{-30pt}  % Left margin
\setlength\evensidemargin{-30pt} % Left margin (even pages) - only relevant with 'twoside' article option
\frenchspacing % Reduces space after periods to make text more compact for a three-column layout

\usepackage{caption}         %
\usepackage{charter}         % Charter font for main content
\usepackage{graphicx}        % Required for including images
\usepackage{amssymb,amsmath} % Math packages
\usepackage{multicol}        % Required for the three-column layout of the document
\usepackage{url}             % Clickable links
\usepackage{enumitem}        % Reduces the amount of space within and between lists with [noitemsep,nolistsep]
\usepackage{marvosym}        % Required for the use of symbols
\usepackage{wrapfig}         % Allows wrapping text around figures
\usepackage[T1]{fontenc}     % Use 8-bit encoding that has 256 glyphs
\usepackage{mathptmx}        % to use Times font
\usepackage{datetime}        % Required for defining a custom date style
\usepackage{fancyhdr}        % Required to define custom headers/footers
\usepackage{setspace}
\pagestyle{fancy}            % Enables the custom headers/footers for all pages following this
\newdateformat{mydate}{\monthname[\THEMONTH] \THEYEAR}          % Set a custom date format
\usepackage[pdfpagemode=FullScreen, colorlinks=false]{hyperref} % Link colors and PDF behavior in Acrobat
\usepackage{hyperref}                                      %
\hypersetup{
    colorlinks = true,
    citecolor  = blue,
    linkcolor  = blue,
    urlcolor   = blue
}


% =====================================================
%              Header and footer
% =====================================================
\lfoot{}   % empty Left footer
% center footer
\cfoot{\footnotesize 
\Mundus\ \href{http://www.ieee-cps.org/}{ieee-cps.org}}
\rfoot{\footnotesize Page \thepage}  % Right footer - page counter
\renewcommand{\headrulewidth}{0.0pt} % No horizontal rule for the header
\renewcommand{\footrulewidth}{0.4pt} % Horizontal rule separating the footer from the document
\fancyhead{}                         % NO Header


%-----------------------------------------------------------
% Creates a horizontal rule
\newcommand{\HorRule}[1]{\noindent\rule{\linewidth}{#1}}
% Creates a shorter separator rule
\newcommand{\SepRule}
{
    \noindent
    \begin{center}
    \rule{250pt}{1pt}      % Page width and rule width
    \end{center}
}


% ================================================
%       Define title and article styles
% ================================================
% ==== Newsletter title
\newcommand{\LetterTitle}[1]
{
    \begin{center}
    \usefont{T1}{ptm}{b}{n}  % Use the Bera Sans Bold font
    \Huge #1
    \end{center}	
    \par \normalsize \normalfont
    \HorRule{3pt}
}

% ==== Print Letter Header
% ==== Input: 1) Volume num, 2) Issue numbers, 3) Date 
\newcommand{\LetterHeader}[3]
{
    \usefont{T1}{ptm}{n}{n}  % Use the Bera Sans Bold font
    \hfill \textsc{Volume #1, Issue #2, #3}  % Right-aligned date and issue number
    \par \normalsize \normalfont
}

% ==== section title
\newcommand{\SectionTitle}[1]
{
    \HorRule{1pt}
    \usefont{T1}{ptm}{b}{n}               % Use Times New Roman Bold font
    \vspace{3pt}\Large #1\vspace{1pt}    % Print the title with space around it in a larger font size
    \par \normalsize \normalfont
    \HorRule{1pt}
}

% ==== article title
\newcommand{\NewsTitle}[1]
{
    \begin{center}
    %\usefont{T1}{fvs}{b}{n}              % Use Bera Sans Normal font
    \usefont{T1}{ptm}{b}{n}               % Use Times New Roman Bold font
    \vspace{14pt}\Large #1\vspace{1pt}    % Print the title with space around it in a larger font size
    \par \normalsize \normalfont
    \end{center}
}

% ==== article authors
\newcommand{\NewsAuthor}[1]
{
    \begin{center}
    %\hfill by \textsc{#1} \vspace{20pt} % Right-aligned author name in small caps with space after it
    \usefont{T1}{ptm}{n}{n}               % Use Times New Roman Bold font
    \textsc{#1} \vspace{4pt} % author name in small caps with space after it
    \par \normalfont
    \end{center}
}

